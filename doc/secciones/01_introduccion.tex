\UseRawInputEncoding
\chapter{Introducción}

Este proyecto es software libre, y está liberado con la licencia \cite{gplv3}.

\section{Motivación}
Debido a la crisis económica del 2008 la sanidad española no ha parado de recibir recortes, puede observarse en este
\href{https://www.consalud.es/politica/decada-recortes-2009-2018-efectos-infrafinanciacion-sanidad_87083_102.html}{artículo} donde
se estudia la financiación de la sanidad desde entonces. Esto provoca que se disminuya el número de sanitarios en España.
Esta situación que se lleva dando desde hace más de una década produce una sobrecarga de trabajo para los sanitarios y un retardo
en todas las listas de esperas, tanto en cirugías como en la atención primaria.
Con la llegada de la pandemia y del covid-19 todo empeoró llegando a colapsar totalmente la atención primaria debido a la demanda
de citas previas y a la escasez de profesionales enfocados a este servicio, ya que, también desde hace años los sanitarios se aglomeran
en los Hospitales por mejores situaciones laborales y salarios. En las siguientes links podemos observar la situación que he descrito.
\begin{itemize}
    \item \url{https://elpais.com/sociedad/2021-02-08/la-pandemia-sume-a-la-atencion-primaria-en-una-saturacion-permanente.html}
    \item \url{https://www.lavanguardia.com/vida/20211225/7952109/atencion-primaria-colapso-pacientes-medico-dias-espera.html}
    \item \url{https://www.lavozdegalicia.es/noticia/galicia/2022/01/16/dejando-atender-casos/0003_202201G16P2993.htm}
\end{itemize}
El problema que trato de abordar se escapa a mis posibilidades, sin embargo, mediante la herramienta que voy a desarrollar
durante el proyecto podré ofrecer una gran ayuda para los sanitarios centrados en la atención primaria.
Mi principal motivación es aportar mediante el uso de sistemas informáticos mi grano de arena en este gran problema con el que
tenemos que lidiar a día de hoy, construyendo una aplicación de citas médicas el cual tiene integrada un sistema de triage.

\section{Objetivos}

\textbf{Crear el triage inteligente de las consultas médicas}. El sistema informático consistirá en un triage de consultas médicas; el triage en el ámbito
 de la sanidad es un proceso que permite gestionar el flujo de los pacientes teniendo en cuenta el riesgo cuando la demanda y necesidades clínicas
 superan los recursos, es muy importante en urgencias clínicas \href{https://scielo.isciii.es/scielo.php?script=sci_arttext&pid=S1137-66272010000200008}{Aquí}
 se puede encontrar más información.

El triage inteligente es un gestor de citas previas médicas organizadas por una heurística, la cual priorizará las citas de los pacientes
con mayores riesgos, evitando las repercusiones negativas de la saturación actual en la sanidad. Sin embargo, no dejará ociosas al resto
de citas con menor prioridad, porque cada riesgo tendrá una franja horaria asociada al día.

La asignación de riesgo se realizará a partir de un formulario que deben rellenar el paciente al registrarse, esta prioridad
se comprobará y podrá ser modificado por los sanitarios en cualquier momento desde el sistema. Esto es necesario, ya que,
la mayoría de pacientes aumentarán de riesgo con los años y con las enfermedades que padezcan.

El triage tiene un gran potencial que no se aprovecha en los sistemas de citas de la sanidad actual, de forma que, con el triage inteligente
voy a demostrar que los tiempos medios de espera por riesgo disminuirán con respecto a otros sistemas de citas como \href{https://www.sspa.juntadeandalucia.es/servicioandaluzdesalud/clicsalud/pages/portada.jsf?caducada=1}{ClicSalud+},
verificando que hay una mejora notable. Esta comprobación se llevará a cabo simulando sobre el triage inteligente una carga de trabajo similar
a la de ClicSalud+ y midiendo tiempos.




