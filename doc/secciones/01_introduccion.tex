\UseRawInputEncoding
\chapter{Introducción}

Este proyecto es software libre, y está liberado con la licencia \cite{gplv3}.

\section{Motivación}
Debido a la crisis económica del 2008 la sanidad española no ha parado de recibir recortes, puede observarse en este
\href{https://www.consalud.es/politica/decada-recortes-2009-2018-efectos-infrafinanciacion-sanidad_87083_102.html}{artículo} donde
se estudia la financiación de la sanidad desde entonces. Esto provoca que se disminuya el número de sanitarios en España.
Esta situación que se lleva dando desde hace más de una década produce una sobrecarga de trabajo para los sanitarios y un retardo
en todas las listas de esperas, tanto en cirugías como en la atención primaria.
Con la llegada de la pandemia y del covid-19 todo empeoró llegando a colapsar totalmente la atención primaria debido a la demanda
de citas previas y a la escasez de profesionales enfocados a este servicio, ya que, también desde hace años los sanitarios se aglomeran
en los Hospitales por mejores situaciones laborales y salarios. En las siguientes links podemos observar la situación que he descrito.
\begin{itemize}
    \item \url{https://elpais.com/sociedad/2021-02-08/la-pandemia-sume-a-la-atencion-primaria-en-una-saturacion-permanente.html}
    \item \url{https://www.lavanguardia.com/vida/20211225/7952109/atencion-primaria-colapso-pacientes-medico-dias-espera.html}
    \item \url{https://www.lavozdegalicia.es/noticia/galicia/2022/01/16/dejando-atender-casos/0003_202201G16P2993.htm}
\end{itemize}
El problema que trato de abordar se escapa a mis posibilidades, sin embargo, mediante la herramienta que voy a desarrollar
durante el proyecto podré ofrecer una gran ayuda para los sanitarios centrados en la atención primaria.
Mi principal motivación es aportar mediante el uso de sistemas informáticos mi grano de arena en este gran problema con el que
tenemos que lidiar a día de hoy, construyendo una aplicación de citas médicas el cual tiene integrada un sistema de triage.

\section{Objetivos}

\textbf{Abstraer el concepto de triage. Automatizarlo y aplicarlo a otros ámbitos}. El sistema informático se basará en un triage de consultas médicas; el triage en el ámbito
 de la sanidad es un proceso que permite gestionar el flujo de los pacientes teniendo en cuenta el riesgo cuando la demanda y necesidades clínicas
 superan los recursos, es muy importante en urgencias clínicas \href{https://scielo.isciii.es/scielo.php?script=sci_arttext&pid=S1137-66272010000200008}{Aquí}
 se puede encontrar más información.

 El objetivo principal consiste en aprovechar el potencial del triage ya existente en urgencias y explotarlo al máximo, de tal forma que
 consigamos aportar una nueva herramienta a los sanitarios con la intención de paliar la saturación de la sanidad en general y sus efectos negativos sobre los pacientes.
 El triage inteligente consistirá en una abstracción del triage que ya se aplica en urgencias y automatizarlo en un sistema informático, incluso aplicarlo
 otros ámbitos de la sanidad para mejorar su eficacia y eficiencia.

 Los objetivos genéricos de un producto software son que pueda desplegarse en la nube, que pueda ser accesible por cualquier dispositivo que acceda a la aplicación web,
 que mantenga la seguridad y privacidad de los datos de los pacientes y sanitarios y que la experiencia de usuario sea lo más satisfactoria posible.

