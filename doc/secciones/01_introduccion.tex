\UseRawInputEncoding
\chapter{Introducción}

Este proyecto es software libre, y está liberado con la licencia \cite{gplv3}.

\section{Motivación}
Debido a la crisis económica del 2008 la sanidad española no ha parado de recibir recortes, puede observarse en este
\href{https://www.consalud.es/politica/decada-recortes-2009-2018-efectos-infrafinanciacion-sanidad_87083_102.html}{artículo} donde
se estudia la financiación de la sanidad desde entonces. Esto provoca que se disminuya el número de sanitarios en España.
Esta situación que se lleva dando desde hace más de una década produce una sobrecarga de trabajo para los sanitarios y un retardo
en todas las listas de esperas, tanto en cirugías como en la atención primaria.
Con la llegada de la pandemia y del covid-19 todo empeoró llegando a colapsar totalmente la atención primaria debido a la demanda
de citas previas y a la escasez de profesionales enfocados a este servicio, ya que, también desde hace años los sanitarios se aglomeran
en los Hospitales por mejores situaciones laborales y salarios. En las siguientes links podemos observar la situación que he descrito.
\begin{itemize}
    \item \url{https://elpais.com/sociedad/2021-02-08/la-pandemia-sume-a-la-atencion-primaria-en-una-saturacion-permanente.html}
    \item \url{https://www.lavanguardia.com/vida/20211225/7952109/atencion-primaria-colapso-pacientes-medico-dias-espera.html}
    \item \url{https://www.lavozdegalicia.es/noticia/galicia/2022/01/16/dejando-atender-casos/0003_202201G16P2993.htm}
\end{itemize}
El problema que trato de abordar se escapa a mis posibilidades, sin embargo, mediante la herramienta que voy a desarrollar
durante el proyecto podré ofrecer una gran ayuda para los sanitarios centrados en la atención primaria.
Mi principal motivación es aportar mediante el uso de sistemas informáticos mi grano de arena en este gran problema con el que
tenemos que lidiar a día de hoy, construyendo una aplicación de citas médicas el cual tiene integrada un sistema de triage.

\section{Objetivos}

\textbf{Crear el triage inteligente de consultas médicas}. El sistema informático se basará en un triage de consultas médicas; el triage en el ámbito
 de la sanidad es un proceso que permite gestionar el flujo de los pacientes teniendo en cuenta el riesgo cuando la demanda y necesidades clínicas
 superan los recursos, es muy importante en urgencias clínicas \href{https://scielo.isciii.es/scielo.php?script=sci_arttext&pid=S1137-66272010000200008}{Aquí}
 se puede encontrar más información. El triage inteligente consistirá en una abstracción del triage que ya se aplica en urgencias y automatizarlo en un sistema informático.

El triage inteligente es un gestor de citas guiadas por una heurística, la cual priorizará las citas de los pacientes
con mayores riesgos, evitando las repercusiones negativas de la saturación actual en la sanidad. Sin embargo, el triage no dejará
ociosa a los pacientes en espera, después de que superar un tiempo máximo de espera, se volverá a evaluar su riesgo teniéndolo en cuenta.

La asignación de riesgo se realizará a partir de un formulario que debe rellenar el paciente antes de solicitar la cita.
Los resultados del formulario pueden ser supervisados por un sanitario y modificar el riesgo asignado al paciente bajo su propio criterio.

Inicialmente, se aplicará el triage inteligente en dos posibles escenarios:
\begin{itemize}
\item \textbf{Urgencias:} En esta situación un paciente que llegue a urgencias deberá completar el formulario y se le asignará un riesgo y un tiempo de espera máximo.
\item \textbf{Citologías:} En esta situación el sanitario solicita al paciente que rellene el formulario, dependiendo de los resultados se le asignará una fecha más o menos próxima.
\end{itemize}

De esta forma podríamos aprovechar el gran potencial del triage automatizarlo para urgencias y aplicarlo a diferentes ámbitos de la sanidad.
La abstracción de este concepto puede llevar a una mejora notable en el diagnóstico precoz de enfermedades donde puede ser crucial para el paciente, como un cáncer.
