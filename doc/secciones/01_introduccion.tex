\chapter{Introducción}

Este proyecto es software libre, y está liberado con la licencia \cite{gplv3}.

\section{Motivación}
Debido a la crisis económica del 2008 la sanidad española no ha parado de recibir recortes desde entonces, puede observarse en este
\href{https://www.consalud.es/politica/decada-recortes-2009-2018-efectos-infrafinanciacion-sanidad_87083_102.html}{artículo} donde
se estudia la financiación de la sanidad desde entonces. Esto provoca que se disminuya el número de sanitarios en España, 
esta situación que se lleva dando desde hace más de una década produce una sobrecarga de trabajo para los sanitarios y un retardo 
en todas las listas de esperas, tanto en cirugias como en la atención primaria. 
Con la llegada de la pandemia y del covid-19 todo empeoró llegando a colapsar totalmente la atención primaria debido a la demanda
de citas previas y a la escasez de profesionales enfocados a este servicio ya que también desde hace años los sanitarios se aglomeran
en los Hospitales por mejores situaciones laborales y salarios. En las siguientes links podemos observar la situación que he descrito.
\begin{itemize}
    \item \url{https://elpais.com/sociedad/2021-02-08/la-pandemia-sume-a-la-atencion-primaria-en-una-saturacion-permanente.html} 
    \item \url{https://www.lavanguardia.com/vida/20211225/7952109/atencion-primaria-colapso-pacientes-medico-dias-espera.html}
    \item \url{https://www.lavozdegalicia.es/noticia/galicia/2022/01/16/dejando-atender-casos/0003_202201G16P2993.htm}

El problema que trato de abordar se escapa a mis posibilidades, sin embargo mediante la herramienta que voy a desarrollar 
durante el proyecto podré aportar una gran ayuda para los sanitarios centrados en la atención primaria.
Mi principal motivación es aportar mediante el uso de sistemas informáticos mi grano de arena en este gran problema con el que 
tenemos que lidiar a día de hoy, construyendo una aplicación de citas médicas el cuál tiene integrado un sistema de triage. 
Además estará centrado para el uso de todas las personas y sanitarios, haciendo incapié en la accesibilidad de la app.

\section{Objetivos}

\subsection{Principal}

\textbf{Crear el triage inteligente de las consultas médicas}. El objetivo se centra en desarrollar un sistema informático que sirva
como apoyo a los sanitarios de atención primaria reduciendo, todo lo posible, la carga de trabajo de ellos, liberando de esta forma 
la saturación en las colas de esperas. Además al tratarse de un triage de consultas priorizaremos las personas con mayor riesgo 
aumentando la eficacia de la atención primaria. Gracias a este sistema podré aportar lo posible a la sanidad desde mi ámbito, la informática

El sistema informática consistirá en un triage de consultas médicas ; el triage en el ámbito de la sanidad es un proceso que permite gestionar
el flujo de los pacientes teniendo en cuenta el riesgo cuando la demanda y necesidades clínicas superan los recursos, es muy importante en urgencias clínicas
\href{https://scielo.isciii.es/scielo.php?script=sci_arttext&pid=S1137-66272010000200008}{Aquí} se puede encontrar más información

El sistema de triage conlleva crear un sistemas de colas de citas priorizada, las cualés se ordenarán mediante una heuristica que tendrá 
en cuenta el riesgo de cada usuario, también debe de estar integrada en una aplicación accesible para todos los usuarios y
asegurando la calidad del código

\subsection{Opcionales}
Los objetivos secundarios podrían ser:

\begin{itemize}
    \item Almacenar el historial médico de una consulta en la aplicación
    \item Modificar el riesgo de los usuario 

\subsection{Proyecto a futuro}
Los objetivos que se quedarán para el futuro del proyecto serán: 

\begin{itemize}
    \item Aplicar la aplicación para consultas especializadas 
    \item Construir un plugin para acceder a las recetas de los medicamentos de una consulta
    \item Crear consultas automatizadas para enfermedades crónicas
