\UseRawInputEncoding
\chapter{Implementación}

A lo largo de este capítulo se pretende reflejar las herramientas usadas durante todo el proceso de desarrollo y
se argumentará por qué se han tomado estas decisiones, al igual que en el capítulo de metodología \ref{ch:metodología} explicamos
la planificación y forma de trabajo.

La implementación del software se ha dividido en hitos. Cada hito representa un producto mínimamente viable que se publicará.
Los \href{https://github.com/RubenDelgadoPareja/TFG-Triage-Inteligente-Consulta-Medica/milestones}{hitos} han sido definidos en GitHub y explicados en la sección \ref{sec:hitos}.
Cada uno de ellos tiene asignados \href{https://github.com/RubenDelgadoPareja/TFG-Triage-Inteligente-Consulta-Medica/issues}{\textit{issues}} que se corresponden
con los distintos problemas que han ido surgiendo durante el desarrollado, cómo los hemos solucionado, qué herramientas se han utilizado y por qué.
La solución de las issues asignadas al hito conforman el llamado producto mínimo viable (\textit{PMV}) del hito.

\section{Hito 0: Infraestructura inicial y documentación}

El PMV del hito 0 es un repositorio inicial formado por una documentación y un conjunto de herramientas configuradas para el desarrollo del proyecto.
En la documentación se ha plasmado la planificación inicial del proyecto y configuraciones de herramientas. En el capítulo de metodología ya definimos el hito \ref{sb:hito0} a alcanzar.
Podemos encontrar dos tipos de herramientas configuradas, dependiendo de su uso, las herramientas de desarrollo y las de documentación.
Aunque inicialmente tengo que aclarar dónde se completa tanto documentación como desarrollo.

\subsection{Entorno de Desarrollo Integrado}
El primer problema es encontrar una herramienta donde comenzar el desarrollo del proyecto y configuraciones de herramientas, para ello necesitamos un entorno de desarrollo integrado (\textit{IDE}).
Las herramientas que cumplen con la necesidad pueden ser Visual Studio Code, Atom, Sublime Text, WebStorm, etc.
Visual Studio Code destaca por su ligereza, multiplataforma, gran cantidad de extensiones y configuraciones, además de ser gratuita, no como WebStorm.
Respecto al resto, Visual Studio Code permite tener una terminal integrada y tiene mayor comunidad. Además, es la herramienta con la que tengo experiencia usándola y me siento cómodo desarrollando.

\subsection{Memoria}
La memoria es muy importante en el desarrollo de un proyecto, ya que es el reflejo de todo el trabajo realizado y es necesario para la evaluación del proyecto.
Debemos de encontrar una herramienta que nos permita escribir la memoria de forma cómoda y eficiente. Además de comprobar constantemente que lo escrito esté correcto.

\paragraph*{Generación de la memoria}
La memoria se ha escrito empleando el lenguaje LaTeX, que es un sistema de composición de documentos bastante empleado
en el ámbito académico y científico, es necesario para poder generar la documentación final, agrupando cada capítulo, sección y
gráfica.

\paragraph*{Corrector de la memoria}
La herramienta \href{https://github.com/sylvainhalle/textidote}{TeXtidote} nos permite corregir la ortografía, gramática y semántica de la documentación escrita. Se puede ejecutar la corrección a través del comando \textit{check} del Makefile,
generando un borrador con la documentación marcando los errores y posibles cambios. Además, se ha añadido este comando a los hooks de GitHub para ejecutarlo antes de cada commit.

Finalmente, la herramienta para gestionar tanto la generación como corrección de la memoria es make. Existe una tarea para cada herramienta, \textit{all} para generar la memoria y \textit{check} para corregirla.

\subsection{Herramientas de desarrollo}

\subsubsection*{Lenguaje de programación}
Primero debemos de escoger un lenguaje de programación para el desarrollo del proyecto.
Existen infinidad de lenguajes de programación como JavaScript, Java, Python, Ruby, C++, C\#, etc.
Sobre toda estas he escogido Typescript junto con Node.js, ya que da una gran robustez gracias al tipado, es un lenguaje moderno y con una gran comunidad.
Cuenta con una gran batería de librerías que agiliza el desarrollo y permite la implementación rápida de características justamente lo que se quiere con el
desarrollo ágil.

\subsubsection*{Análisis estático}
En TypeScript a pesar de tener un tipado fuerte, es posible que se cometan errores que no se detecten hasta que se ejecute el código, por ello es necesario un análisis estático del código.
Para el desarrollo ágil es inprescindible garantizar la calidad, la eficiencia y sostenibilidad del código a largo plazo.
Las posibles herramientas a escoger son SonarQube, ESLint, TSLint, Checkmarx, PMD. Sin embargo, ESLint es la única herramienta especifica para el lenguaje escogido, TypeScript.
Mientras que SonarQube es una herramienta más generalista y Checkmarx es una herramienta de pago. PMD es una herramienta que se utiliza para Java, por lo que no es aplicable.
ESLint te permite aplicar reglas personalizadas, reglas predefinidas y configuraciones de estilo para garantizar que tu código sea consistente y libre de problemas.
No se ha escogido TSLint, por el hecho de que, está obsoleta y no tiene mantenimiento.

\subsubsection*{Integración Continua}
La integración continua es una práctica de desarrollo de software que tiene como objetivo principal mejorar la calidad del código, aumentar la eficiencia y agilizar el proceso de entrega de software.
Las diferentes opciones que encontramos para el cliente o local pueden ser git, SVN, Mercurial y para el servidor donde se harán las pruebas y la integración se presentan las siguientes
alternativas Jenkins, Travis, Circle, GitHub Actions. He decido escoger git para el cliente y GitHub Actions para el servidor porque me resultan fáciles de configurar y escalabilidad.
Gracias a esta metodología podemos automatizar la ejecución de test, linter y corrector ortográfico, de esta manera comprobaremos antes de ciertas acciones con los hooks de GitHub que todo está
correctamente. También se había barajado la opción de utilizar un paquete llamado \textit{husky} para Node.js, pero no se ha escogido finalmente para aprender nuevas tecnologías.
Con esta herramienta se ha configurado un workflow que se ejecuta en cada push a la rama principal, en el que se ejecutan los test, el linter y el corrector ortográfico, ya mencionado en la sección de control de calidad \ref{sc:control-calidad}.
