\UseRawInputEncoding
\chapter{Implementación}

En este capítulo se pretende reflejar las herramientas usadas durante toda la implementación y
se argumentará por qué se han tomado estas decisiones, al igual que en el capítulo de metodología \ref{ch:metodología} explicamos
la planificación y forma de trabajo.

El desarrollo del software se ha dividido en hitos \ref*{sec:hitos}. Cada hito representa un PMV que se publicará.
PMV son las siglas de Producto Mínimo Viable, es la versión más simple y básico del producto que proporciona valor al cliente.

Los \href{https://github.com/RubenDelgadoPareja/TFG-Triage-Inteligente-Consulta-Medica/milestones}{hitos} han sido definidos en GitHub.
Cada uno de ellos tiene asignados \href{https://github.com/RubenDelgadoPareja/TFG-Triage-Inteligente-Consulta-Medica/issues}{\textit{issues}} que se corresponden
con los distintos problemas que han ido surgiendo durante el desarrollado, cómo los hemos solucionado, qué herramientas se han utilizado y por qué.
La solución de las issues asignadas al hito conforman el llamado producto mínimo viable (\textit{PMV}) del hito.

\section{Hito 0: Infraestructura y documentación inicial}

Lo he llamado hito número 0 porque es el punto de partida del proyecto y no aporta un valor real al cliente, sino a nosotros como desarrolladores.
El PMV del hito 0 \ref{sb:hito0} es un repositorio inicial formado por una documentación y un conjunto de herramientas configuradas para el desarrollo del proyecto.
Las herramientas configuradas durante este hito son las siguientes:

\subsection{Entorno de desarrollo}
Antes de poder empezar con el hito 0 debemos de plantearnos qué entorno usar como base de todo el trabajo.
Necesitamos una herramienta donde comenzar el desarrollo del proyecto y configuraciones de herramientas.
Los criterios de búsqueda de este entorno será el estándar, el software libre, multilenguaje, de calidad, de fácil empleo y configuración, por orden de prioridad.
Después de una pequeña búsqueda sobre los entornos de desarrollo, no existe un estándar, pero sí que hay un grupo de más usados, los cuales son: VS Code, IntelliJ IDEA, PyCharm, Eclipse, Netbeans.
De esta lista he seleccionado finalmente VS Code, es el único que cumple con la mayoría de criterios. La mayoría de entornos depende del lenguaje de programación, pero al no haberlo
decidido aún elijo VS Code, además es software libre.

\subsection{Integración continua}
La integración continua es una práctica de desarrollo de software que tiene como objetivo principal mejorar la calidad del código, aumentar la eficiencia y agilizar el proceso de entrega de software.
Debemos encontrar una herramienta que nos permita automatizar los procesos de integración en la nube, teniendo en cuenta que sea gratuita para código abierto, fácil de configurar y sea compatible con el repositorio del proyecto en GitHub.
Las herramientas que cumplen con estos requisitos son Travis CI, CircleCI, GitHub Actions. La opción elegida es GitHub Actions, es la herramienta perfecta en este caso, ya que es gratuita, fácil de configurar y está integrada en el propio GitHub.
Esto nos ayuda a poder relanzar los flujos de trabajo desde la propia interfaz de GitHub.

Gracias a esta metodología podemos automatizar la ejecución de herramientas de calidad, de esta manera comprobaremos antes de ciertas acciones con los hooks de GitHub que todo está correctamente.
La potencia de esta herramienta es ejecutar diferentes flujos de trabajo en paralelo o \textit{workflow} en el que hacen diferentes comprobaciones garantizando la calidad en todo momento.
Cada workflow se ejecuta en cada push a la rama principal, estos workflow podrán ejecutar diferentes tareas dependiendo de la configuración.
A continuación sigo con las herramientas que aportan calidad tanto al código como a la documentación.

\subsection{Herramientas de documentación}
La memoria es muy importante en el desarrollo de un proyecto, ya que es el reflejo de todo el trabajo realizado y es necesario para la evaluación del proyecto.
Debemos de encontrar una herramienta que nos permita escribir la memoria de forma cómoda y eficiente. Además de revisar constantemente que lo escrito esté correcto.

\subsubsection{Generación de la memoria}
La memoria se ha escrito empleando el lenguaje LaTeX, que es un sistema de composición de documentos bastante empleado
en el ámbito académico y científico, es necesario para poder generar la documentación final, agrupando cada capítulo, sección y
gráfica. Para facilitar la construcción de la memoria se ha utilizado un gestor de tareas con la finalidad de automatizar la generación de la memoria.
Las opciones que se han tenido en cuenta son Make, CMake, Gradle, Rake o Gulp. Se ha escogido Make porque es la más sencilla de configurar y la más extendida, el resto
están ligados a un lenguaje de programación en concreto y no se ajustan a las necesidades del proyecto, ya que añadiría una complejidad innecesaria.

\subsubsection{Corrector de la memoria}
Para garantizar la calidad de la documentación dentro del desarrollo ágil, es necesario un corrector ortográfico y gramatical.
La herramienta \href{https://github.com/sylvainhalle/textidote}{TeXtidote} nos permite corregir la ortografía, gramática y semántica de la documentación escrita.
Se puede ejecutar la corrección a través del comando \textit{check} del Makefile, generando un borrador con la documentación marcando los errores y posibles cambios.
Además, se ha añadido este comando a los hooks de GitHub para ejecutarlo antes de cada commit.

Con la intención de continuar con el paradigma de la calidad y las buenas prácticas claves para el desarrollo ágil, se ha configurado el entorno con
una tarea en el Makefile para ejecutar para cada herramienta, \textit{all} para generar la memoria y \textit{check} para revisarla.

\subsection{Herramientas de desarrollo}
El código fuente debe de garantizar la calidad, la eficiencia y mantenimiento del código a largo plazo. Para ello necesitamos de la ayuda de diferentes herramientas
que nos ayuden a detectar errores durante y después el desarrollo. Pero estas herramientas dependen del lenguaje de programación escogido, por lo que debemos de definirlo antes.

\subsubsection{Lenguaje de programación}
Primero debemos de escoger un lenguaje de programación para el desarrollo del proyecto, el cual cumpla una serie de criterios para que sea adecuado.
Para elegir correctamente deberíamos de hacer un estudio del alcance y objetivos del proyecto, los recursos y habilidades disponibles, la experiencia del desarrollador etc.
Los criterios que para elegir el lenguaje son la experiencia del programador, la portabilidad, legibilidad, seguridad del código y la comunidad.
Los lenguajes que cumplen estos criterios son los siguientes: JavaScript, TypeScript, Java, Ruby, C++, Python. Inicialmente, porque son los lenguajes con los que he trabajado.
Como el proyecto será una aplicación web buscamos lenguajes portables con los diferentes navegadores actuales, destaca JavaScript porque los navegadores desde 2012 soportan ECMAScript 5.1 que está basado en JavaScript.
Además, como TypeScript es un superconjunto de JavaScript, también sería buena opción en ese aspecto y gracias al tipado fuerte que tiene aporta seguridad y legibilidad al código, mejorando la calidad del software.
Los lenguajes como Java y Python también serían buenas opciones, pero finalmente se han descartado por la poca experiencia y la falta de recursos para comenzar a desarrollar en ellos.
Por tanto, el lenguaje de programación escogido es TypeScript;

\subsubsection{Análisis estático}
Para el desarrollo ágil es imprescindible garantizar la calidad, la eficiencia y sostenibilidad del código a largo plazo.
En TypeScript, a pesar de tener un tipado fuerte, es posible que se cometan errores que no se detecten hasta que se ejecute el código; por ello es necesario un análisis estático del código.
Buscamos una herramienta rápida y sencilla de configurar que detecte errores de programación en TypeScript.
Las posibles herramientas a escoger son ESLint o TSLint, porque son las únicas herramientas especifica para el lenguaje escogido, TypeScript. Aunque, TSLint está obsoleta y no tiene mantenimiento.
Nos quedamos con ESLint que te permite aplicar reglas personalizadas, reglas predefinidas y configuraciones de estilo para garantizar que tu código sea consistente y libre de problemas.

\subsubsection{Pruebas unitarias}
