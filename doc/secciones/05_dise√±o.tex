\UseRawInputEncoding
\chapter{Diseño Dirigido por el Dominio}

Para el diseño del software he elegido aplicar DDD\textit{(Domain Driven Design)} para enfocar el
desarrollo del producto en el dominio del problema, de forma que, el sistema contemple el lenguaje y reglas
del dominio. Este tipo de diseño se centra en el usuario al igual que la metodología ágil, la cual,
también se usa en el proyecto. El potencial de este diseño reside en que se alinea estrechamente el producto
con los conceptos y términos clave del dominio. Sin embargo, tiene algunos inconvenientes para un TFG como
necesitar una comunicación continua e iterativa con expertos del dominio.

\section{Lenguaje Ubicuo}
He creado un glosario con un conjunto de términos pertenecientes al dominio que se emplearán dentro del desarrollo del producto,
de esta manera cuando se den las reuniones entre los desarrolladores y los expertos de dominios no dará lugar a confusiones,
ya que, habremos definido con anterioridad las palabras claves.

Estas son algunas definiciones que ayudarán a establecer un lenguaje común entre los expertos de dominio y los
desarrolladores al discutir y codificar las reglas de negocio.

\begin{itemize}
\item \textbf{Cita:} Una cita se refiere a una programación de tiempo para que un paciente se encuentre con un médico o profesional de atención primaria en un centro de salud.
\item \textbf{Paciente:} Una persona que busca atención médica en el sistema de atención primaria.
\item \textbf{Médico:} Un proveedor de atención médica que ofrece servicios de atención primaria, como médicos generales o especialistas en medicina familiar.
\item \textbf{Riesgo:} Un valor asociado al cálculo de diferentes factores que implica un mayor peligro cuando alguien enferma
\item \textbf{Centro de salud:} Un sitio físico donde se brinda atención primaria a los pacientes, como un consultorio médico, una clínica o un centro de atención ambulatoria.
\item \textbf{Historia clínica:} Un registro médico que contiene información detallada sobre la salud de un paciente, incluyendo diagnósticos, tratamientos anteriores, alergias y otros datos relevantes.
\item \textbf{Síntomas:} Los signos o manifestaciones físicas que indican una posible enfermedad o condición médica.
\item \textbf{Diagnóstico:} La identificación de una enfermedad o condición médica basada en los síntomas del paciente, exámenes médicos y evaluaciones clínicas.
\item \textbf{Tratamiento:} Las acciones médicas o terapias recomendadas para abordar una enfermedad o condición médica específica.
\item \textbf{Receta:} Una prescripción médica para medicamentos o terapias específicas para ser administradas al paciente.
\item \textbf{Expediente médico:} Un conjunto de registros y documentos relacionados con la atención médica de un paciente, incluyendo historias clínicas, resultados de exámenes, informes de consultas y otros datos importantes.
\end{itemize}

Dentro del Diseño Dirigido por el Dominio debemos abstraer distintos módulos para disgregar el problema en pequeñas partes
y componerlo desde los trozos más pequeños posibles de código hasta formar el diseño que resuelva este problema.
Por esto es necesario dividir en diferentes secciones cada uno de estos módulos y explicarlos por separado.

\section*{Objetos de Valor}

Son un objeto inmutable, usado para representar un valor. Pueden compararse con otros objetos de valor para comprobar
si tienen los mismos valores representados y no tienen identidad propia.
Suelen ser la combinación de atributos primitivos que adquieren un significado al leerse juntos, por ello deben encapsularse
en un único objeto de valor, por ejemplo las coordenadas geográficas (latitud y longitud). A veces tiene sentido convertir
un atributo primitivo a objeto de valor, dependiendo del dominio del problema.
Los Objetos de Valor identificados son los siguientes: DNI, N.º de seguridad social, Riesgo, Coordenadas, Código Postal.


\section*{Entidades}
Las entidades en DDD se definen por su identidad, lo que significa que son distintas y reconocibles a lo largo del tiempo,
independientemente de los cambios en sus atributos. Por ejemplo, si estás construyendo un sistema de gestión de bibliotecas,
una entidad podría ser un libro. Aunque cambien su estado, título o ubicación, sigue siendo el mismo libro con la misma identidad.

En DDD, las entidades se modelan como objetos que encapsulan su estado y comportamiento relacionados. Por lo tanto,
las entidades no solo tienen atributos, sino que también tienen métodos o funciones que definen cómo interactúan y
responden en el dominio. Estos métodos pueden reflejar las operaciones que se realizan en las entidades, como calcular
el costo total de una orden o reservar un libro en una biblioteca.

El diseño de las entidades en DDD se enfoca en capturar el conocimiento del dominio y representar las reglas y
restricciones del problema de manera adecuada. Se busca construir un modelo rico y expresivo que refleje de cerca
la realidad del dominio y permita una implementación más eficiente y mantenible.
Las Entidades identificadas son los siguientes: Paciente, Médico, Consultorio, Historial Médico.



