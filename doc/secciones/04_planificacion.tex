\chapter{Planificación}

Para poder hablar de la implementación primero debemos de establecer cuál es la metodología y el control de calidad 
de la solución propuesta

\section{Metodología}

Buscamos un tipo de metodología que se centre en el usuario y nos aporte retroalimentación.
La metodología ágil nos permite adaptarnos al cambio de requisitios haciendo uso de las iteraciones y 
evitando caer en una documentación excesiva llena de diagramas y requisitos innecesarios.
La metodología ágil construye el sistema por medio de aportaciones frecuentes de código que aportan valor 
al usuario, es necesario organizar las funcionalidades en bloques y que los cambios siempre surgen de una 
necesidad del usuario
He de mencionar que todos los commits deben de hacer referencia a las tareas preestablecidas, por lo tanto 
podremos saber cuando leyamos el código hacia que historia de usuario o issue se refiere

\subsection{Control de versiones}
Para el control de versiones utilizo git y GitHub como repositorio para hacer un seguimiento de todas las tareas
y poder observar el desarrollo en conjunto del proyecto 

\section{Hitos}
Los hitos pueden definirse como metas a las que debemos de llegar durante el desarrollo del proyecto.
Los hitos son una buena forma de limitar los bloques de trabajo del desarrollo y su principal intención es que 
al final de cada hito consiguamos un producto mínimamente viable (\textit{PMV}) para que sea independiente e iterativo 
Los hitos que se deben de cumplir están reflejados en el GitHub y son los siguientes: 
\subsection*{Hito 0: Infraestructura inicial y documentación}

El hito consiste en alcanzar un repositorio inicial donde compenzar a forjar toda la arquitectura del proyecto.
Además de planificar todos los procesos necesarios para llegar al producto final. Para aceptar el hito se necesita completar:

\begin{itemize}
    \item{Hitos.}
    \item{Historias de Usuario.}
    \item{Comprobadores de gramática y ortografía.}
    \item{Comprobadores de compilación de Latex para la documentación.}
    \item{Creación y configuración del contenedor del proyecto (Docker)}
    \item{Documentar las secciones de Motivación, Objetivos y Metodología}
\end{itemize}

\subsection*{Hito 1: Arquitectura básica del Sistema de colas}

La funcionalidad principal que se va abordar es el sistema de colas que va almacenar las citas solicitadas por los usuarios.
La intención es alcanzar un sistema modularizado que permita almacenar citas en colas con prioridad y reordenarlas con una heurística.
Por lo que para aceptar el PVM debemos de completar: 

\begin{itemize}
    \item{Creacion de personajes y escenarios verosímiles}
    \item{Diseño de un diagrama de clases de citas y colas.}
    \item{Diseñar la heurística de las colas con prioridad.}
    \item{Desarrollar las clases y funciones principales de citas y colas .}
    \item{Test unitarios de cada clase}    
\end{itemize}

\subsection*{Hito 2: Sistema de usuario y riesgos asignados}

La funcionalidad a desarrollar durante este hito se centrará en crear el sistema de usuario que harán uso del producto final.
Sin embargo, aún no se integrará al hito anterior, debemos de asegurarnos que es modular. Además tenemos que establecer la 
jerarquía de privilegios y la gestión de las sesiones. 
Para aceptar este PVM necesitamos: 

\begin{itemize}
    \item{Diseño de un diagrama de clases de usuarios y sesiones.}
    \item{Desarrollar las clases para cada usuario y las funciones que puedan realizar como paciente y sanitario}
    \item{Gestión de sesiones}
    \item{Diseño de vistas de inicio de sesion y perfil}
    \item{Test unitarios de cada clase}    
\end{itemize}

\subsection*{Hito 3: Integración de clases y evaluación}

Finalmente este hito pretende agrupar todo lo desarrollado anteriormente y hacerlo funcionar de forma modular. 
Después de todo esto se realizará un estudio de resultados y prestaciones, al igual que finalizar los apartados 
restantes de la documentación. 
Para aceptar este PVM se deberá realizar: 

\begin{itemize}
    \item{Diseño de la interfaz de usuario integrando todas las funcionalidades de cada clase.}
    \item{Test de integración del Triage}
    \item{Análisis de los resultados y prestaciones}
    \item{Documentar los apartados de Estado del Arte, Conclusión y Trabajo a Futuro}
\end{itemize}

\section{Historias de Usuarios}
Las historias de usuario son la unidad mínima de funcionalidad que el usuario necesita del sistema, es decir, 
expresa lo que quiere el usuario del sistema por esto lo hace tan importante, se centra completamente en el 
usuario. Encontraremos las Historias de Usuario repartidas por los Hitos del desarrollo del software 

\section{Issues}
Las Issues son simples tareas que hay que realizar para completar los Hitos. 
Una historia de Usuario podría ser un conjunto de Sssues , aunque en GitHub las he englobado en una sola Issue
Todas las Issues que solucionaré durante el desarrollo para construir el sistema están descritas en GitHub 


\section{Control de calidad}



\section{Estimación de costes}
