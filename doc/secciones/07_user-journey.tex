\UseRawInputEncoding
\section{\textit{User Journeys}}\label{anexo}

\subsection{Rafael González Pérez}
\begin{itemize}
    \item \textbf{Contexto de uso: }
    \begin{itemize}
        \item \textbf{¿Cuándo utiliza el ordenador?: }  Utiliza el ordenador con frecuencia
        \item \textbf{¿Dónde utiliza el ordenador?: } En el trabajo y en casa
        \item \textbf{¿Qué tipo de ordenador utiliza?: } El ordenador de la consulta de urgencias
    \end{itemize}
    \item \textbf{Misión: }
    \begin{itemize}
        \item \textbf{¿Para qué quiere utilizar nuestra aplicación?: } Como sanitario de urgencia quiero priorizar los pacientes con mayor riesgo
        \item \textbf{¿Qué espera encontrar en ella?: } Una aplicación web con el triage de urgencias que organice los turnos de los pacientes
    \end{itemize}
    \item \textbf{Motivación: }
    \begin{itemize}
        \item \textbf{¿Para cuándo quiere utilizarla?: } Cuando comience su jornada laboral en el hospital
        \item \textbf{¿Por qué quiere alcanzar ese objetivo?: } Para automatizar el proceso del triage
    \end{itemize}
    \item \textbf{Actitud hacia la tecnología: } Se siente cómodo navegando por Internet y usando las nuevas tecnologías. No le es ningún impedimento
\end{itemize}

\subsection{Escenario 1 de Rafael González Pérez}
    Rafael es un sanitario que trabaja de guardias en un hospital de urgencias. Rafael se encarga
    de gestionar los pacientes que llegan a urgencias y a través de una pequeña entrevista con el paciente asignarle
    una prioridad u orden de consulta. Rafael descubre que en otros hospitales utilizan un sistema de triage inteligente
    que les ayuda a priorizar a los pacientes y a organizar los turnos de los pacientes. Rafael propone implantarlo en su hospital.
    Rafa consigue implantar el nuevo triage inteligente y lo pone en práctica en su jornada laboral. Los pacientes que van llegando
    a urgencias realizan el cuestionario del triage y automáticamente se les asigna un nivel de prioridad y un tiempo de demora estimado.
    Él puede ver en su pantalla los pacientes que están en espera y el tiempo estimado que les queda para ser atendidos. En caso
    de que sea necesario, Rafa puede asignar manualmente un nivel de prioridad a un paciente en concreto bajo su propio criterio.
    Gracias al triage inteligente, puede organizar los turnos de los pacientes de forma eficiente y rápida.


\subsection{Azucena Rodríguez Peralta}
\begin{itemize}
    \item \textbf{Contexto de uso: }
    \begin{itemize}
        \item \textbf{¿Cuándo utiliza el ordenador?: }  Usa el ordenador a diario
        \item \textbf{¿Dónde utiliza el ordenador?: } En casa y en el trabajo
        \item \textbf{¿Qué tipo de ordenador utiliza?: } El ordenador que le ofrecen en la consulta
    \end{itemize}
    \item \textbf{Misión: }
    \begin{itemize}
        \item \textbf{¿Para qué quiere utilizar nuestra aplicación?: } Para citar con antelación a los pacientes con mayor posibilidad de diagnóstico.
        \item \textbf{¿Qué espera encontrar en ella?: } Una web sencilla para saber cuando tiene cita con sus pacientes y los riesgos / síntomas que presentan.
    \end{itemize}
    \item \textbf{Motivación: }
    \begin{itemize}
        \item \textbf{¿Para cuándo quiere utilizarla?: } Para el trabajo
        \item \textbf{¿Por qué quiere alcanzar ese objetivo?: } Porque quiere diagnosticar posibles enfermedades lo antes posible
    \end{itemize}
    \item \textbf{Actitud hacia la tecnología: } Está muy acostumbrada a usar un ordenador
\end{itemize}

\subsection{Escenario Azucena Rodríguez Peralta}
    Azucena es una mujer que ha estudiado la carrera de Enfermería y está especializada en matrona.
    Acaba de mudarse a una nueva ciudad, donde comenzará a trabajar como matrona de consulta. Azucena se encarga
    de las consultas a cerca del cáncer de cuello de útero donde se presupone un paciente sano, por lo que cita
    a sus pacientes con bastante demora. Con frecuencia, Azucena tiene que atender a pacientes que muestran sintomas
    alarmantes que dan indicios al diagnóstico del cáncer, por lo que propone usar un nuevo sistema de triage inteligente.
    La web le permite a Azucena organizar las citas de sus pacientes y priorizar a los pacientes con mayor riesgo.
    Cuando un paciente recibe una cita con Azucena, este tiene que rellenar un cuestionario con sus síntomas alarmantes.
    Los pacientes con síntomas preocupantes son priorizados y citados con antelación. En caso de un diagnóstico precoz,
    Azucena podrá manualmente cambiar el riesgo del paciente para no tenga tanta demora como los pacientes sanos.
    Gracias al triage inteligente, Azucena puede llegar a diagnosticar enfermedades de forma precoz y rápida.
