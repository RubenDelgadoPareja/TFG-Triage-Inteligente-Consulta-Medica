\section{\textit{User Journeys}}\label{anexo}

\subsection{Samuel Tena Sánchez}
\begin{itemize}
    \item \textbf{Edad: } 24 años
    \item \textbf{Sexo: } Hombre
    \item \textbf{Estado civil: } Soltero
    \item \textbf{Estudia: } Grado Superiro en Artes Plásticas y Diseño
    \item \textbf{Contexto de uso: } 
    \begin{itemize}
        \item \textbf{¿Cuándo utiliza el ordenador?: }  Utiliza el ordenador con frecuencia
        \item \textbf{¿Dónde utiliza el ordenador?: } En casa y en clase 
        \item \textbf{¿Qué tipo de ordenador utiliza?: } Un portatil común de ofimática
    \end{itemize}
    \item \textbf{Misión: } 
    \begin{itemize}
        \item \textbf{¿Para qué quiere utilizar nuestra aplicación?: } Quiere pedir una cita al médico
        \item \textbf{¿Qué espera encontrar en ella?: } Espera encontrar una cita lo antes posible 
    \end{itemize}
    \item \textbf{Motivación: } 
    \begin{itemize}
        \item \textbf{¿Para cuando quiere utilizarla?: } Lo antes posible
        \item \textbf{¿Por qué quiere alcanzar ese objetivo?: } Porque se encuentra mal desde hace días y posiblemente haya enfermado
    \end{itemize}
    \item \textbf{Actitud hacia la tecnología: } Se siente cómodo navegando por Internet y usando las nuevas tecnologías. No le es ningún impedimento
\end{itemize}

\subsection{Escenario 1 de Samuel Tena} 
    Samuel es un chico que estudia un grado superiro en cerámica y después del primer
    cuatrimestre del curso ha sacado muy buenas notas, por lo que su familia ha planeado
    regalarle un viaje a Francia por Navidad. Sin embargo ha comenzado a sentirse muy fatigado
     y necesita pedir cita al médico. La familia de Samuel recientemente
    se ha cambiado a la sanidad privada por su innovador sistema de citas, por lo que Samuel se
    dispone a pedir cita. Accede desde el portal web donde necesita registrarse para pedir cita,
    durante el registro tiene que contestar una serie de preguntas sobre su salud y su historial
    médico para que el sistema sea consciente del estado de su salud actualmente.
    Una vez se registra accede a la página principal de la web donde puede solicitar citas, Samuel
    puede elegir del calendario los días señalados y termina pidiendo la cita para el primer día que
    tiene disponible. Finalmente llega la fecha y asiste al centro de salud, le recetan los
    medicamentos para poder tratarse antes de su viaje.


\subsection{Escenario 2 de Samuel Tena}
    Samuel está en sus vacaciones de verano y se acerca la fecha de su cumpleaños. 
    Después de un fin de semana de mucha fiesta, tiene planeado celebrar su cumpleaños 
    en la playa junto con todos sus amigos. Sin embargo, tiene el estómago muy revuelto 
    y se marea con facilidad justo unos días antes de la celebración, por lo que se dispone 
    a pedir cita al médico para lo antes posible. Inicia sesión en la web con sus credenciales 
    aunque no hay citas disponibles para antes de ese día. Entonces decide tomar la decisión 
    de aplazar su cumpleaños para la próxima semana e ir al médico en la fecha más próxima.
    Después de ir al médico le recentan lo necesario para mejorar y poder celebrar correctamente 
    su cumpleaños. 
    

\subsection{Azucena Rodriguez Peralta}
\begin{itemize}
    \item \textbf{Edad: } 30 años
    \item \textbf{Sexo: } Mujer
    \item \textbf{Estado civil: } Soltera
    \item \textbf{Estudia: } Grado en Medicina
    \item \textbf{Contexto de uso: } 
    \begin{itemize}
        \item \textbf{¿Cuándo utiliza el ordenador?: }  Usa el ordenador a diario
        \item \textbf{¿Dónde utiliza el ordenador?: } En casa y en el trabajo 
        \item \textbf{¿Qué tipo de ordenador utiliza?: } El ordenador que le ofrecen en la consulta
    \end{itemize}
    \item \textbf{Misión: } 
    \begin{itemize}
        \item \textbf{¿Para qué quiere utilizar nuestra aplicación?: } Necesita establecer el riesgo de un paciente
        \item \textbf{¿Qué espera encontrar en ella?: } Espera encontrar el historial médico del paciente y una interfaz sencilla para establecer el riesgo
    \end{itemize}
    \item \textbf{Motivación: } 
    \begin{itemize}
        \item \textbf{¿Para cuando quiere utilizarla?: } Para el trabajo 
        \item \textbf{¿Por qué quiere alcanzar ese objetivo?: } Porque es una tarea de su trabajo
    \end{itemize}
    \item \textbf{Actitud hacia la tecnología: } Está muy acostumbrada a usar las nuevas tecnologías
\end{itemize}

\subsection{Escenario Azucena Rodriguez}
    Azucena trabaja para una empresa de sanidad privada que acaba de adquirir un sistema de triage 
    inteligente para gestionar las colas de citas de atención primaria. De forma que debe de completar
    la tarea de asignar el riesgo a los pacientes que hayan pedido cita, haciendo uso del historial médico
    y datos del paciente. Para ello Azucena ingresa en el sistema con sus datos y accede como sanitario, 
    de esta forma puede modificar y añadir el riesgo a cada paciente pertinente. Azucena observa en la lista
    de pacientes registrados y les otorga un riesgo a los necesarios cumpliendo con su trabajo. 
    Finalmente estos pacientes podrán pedir cita médica cuando lo necesiten