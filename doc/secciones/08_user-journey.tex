\UseRawInputEncoding
\section{\textit{User Journeys}}\label{anexo}

\subsection{Samuel Tena Sánchez}
\begin{itemize}
    \item \textbf{Contexto de uso: }
    \begin{itemize}
        \item \textbf{¿Cuándo utiliza el ordenador?: }  Utiliza el ordenador con frecuencia
        \item \textbf{¿Dónde utiliza el ordenador?: } En casa y en clase
        \item \textbf{¿Qué tipo de ordenador utiliza?: } Un portátil común de ofimática
    \end{itemize}
    \item \textbf{Misión: }
    \begin{itemize}
        \item \textbf{¿Para qué quiere utilizar nuestra aplicación?: } Quiere pedir una cita al médico
        \item \textbf{¿Qué espera encontrar en ella?: } Espera encontrar una cita lo antes posible
    \end{itemize}
    \item \textbf{Motivación: }
    \begin{itemize}
        \item \textbf{¿Para cuándo quiere utilizarla?: } Lo antes posible
        \item \textbf{¿Por qué quiere alcanzar ese objetivo?: } Porque se encuentra mal desde hace días y posiblemente haya enfermado
    \end{itemize}
    \item \textbf{Actitud hacia la tecnología: } Se siente cómodo navegando por Internet y usando las nuevas tecnologías. No le es ningún impedimento
\end{itemize}

\subsection{Escenario 1 de Samuel Tena}
    Samuel es un chico que estudia un grado superior en cerámica y después del primer
    cuatrimestre del curso ha sacado muy buenas notas, por lo que su familia ha planeado
    regalarle un viaje a Francia por Navidad. Sin embargo, ha comenzado a sentirse muy fatigado
     y necesita pedir cita al médico. La familia de Samuel recientemente
    se ha cambiado a la sanidad privada por su innovador sistema de citas, por lo que Samuel se
    dispone a pedir cita. Accede desde el portal web donde necesita registrarse para pedir cita,
    durante el registro tiene que contestar una serie de preguntas sobre su salud y su historial
    médico para que el sistema sea consciente del estado de su salud actualmente. Una vez se registra
    accede a la página principal de la web donde puede solicitar citas, Samuel puede elegir del calendario
    los días señalados y termina pidiendo la cita para el primer día que aparece disponible. La cita disponible
    es para la semana siguiente, ya que, dependiendo de su riesgo no necesita urgencia al ser una persona joven y sin problemas de salud.
    Sin embargo, Samuel lo entiende porque conoce casos más necesitados que el suyo, sobre todo después del covid-19
    Finalmente, llega la fecha y asiste al centro de salud, le recetan los medicamentos para poder tratarse antes de su viaje.


\subsection{Azucena Rodríguez Peralta}
\begin{itemize}
    \item \textbf{Contexto de uso: }
    \begin{itemize}
        \item \textbf{¿Cuándo utiliza el ordenador?: }  Usa el ordenador a diario
        \item \textbf{¿Dónde utiliza el ordenador?: } En casa y en el trabajo
        \item \textbf{¿Qué tipo de ordenador utiliza?: } El ordenador que le ofrecen en la consulta
    \end{itemize}
    \item \textbf{Misión: }
    \begin{itemize}
        \item \textbf{¿Para qué quiere utilizar nuestra aplicación?: } Necesita saber que citas tiene durante el día
        \item \textbf{¿Qué espera encontrar en ella?: } Una web sencilla para saber conocer sus citas asignadas y poder modificar el riesgo del paciente
    \end{itemize}
    \item \textbf{Motivación: }
    \begin{itemize}
        \item \textbf{¿Para cuándo quiere utilizarla?: } Para el trabajo
        \item \textbf{¿Por qué quiere alcanzar ese objetivo?: } Porque es una tarea de su trabajo
    \end{itemize}
    \item \textbf{Actitud hacia la tecnología: } Está muy acostumbrada a usar las nuevas tecnologías
\end{itemize}

\subsection{Escenario Azucena Rodríguez}
    Azucena es una mujer que ha estudiado la carrera de Medicina y está especializada para la atención primaria.
    La acaban de contratar en una empresa de sanidad privada para que se encargue de algunas consultas y le
    han comentado que utilice la aplicación de la empresa para organizarse y gestionar las citas.
    Azucena ingresa con sus credenciales y accede como sanitario, la primera pantalla que encuentra
    es un calendario con todas sus citas asignadas y los pacientes a los que atenderá.
    Azucena esta mañana le ha tocado atender las citas de mayor prioridad, entre los perfiles de los pacientes
    puede observar personas muy mayores, con enfermedades crónicas o gente con un alto riesgo en su salud los cuales
    una cita antes de tiempo puede ser significativa para su estado de salud.
    Azucena después de tener consultar a un paciente decide que el su riesgo debe ser modificado, por lo que
    desde el sistema accede al perfil del paciente y le sube el riesgo, ya que, le han diagnosticado epilepsia
    crónica. Gracias al sistema de triage inteligente Azucena puede cumplir con su trabajo cómodamente.
