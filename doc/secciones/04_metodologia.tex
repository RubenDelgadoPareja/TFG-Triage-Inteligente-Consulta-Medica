\UseRawInputEncoding
\chapter{Metodología}

Para comenzar el desarrollo de un proyecto primero necesitamos una planificación inicial de la que partir.

\section{Planificación}

La planificación que se va a utilizar está basada en los {\href{https://agilemanifesto.org/iso/es/principles.html}{principios del manifiesto ágil}}
Buscamos un tipo de planificación que se centre en el usuario, para que el propio usuario nos aporte retroalimentación.
Una planificación ágil nos permite adaptarnos al cambio de requisitos haciendo uso de las iteraciones y
evitando caer en una documentación excesiva llena de diagramas y requisitos innecesarios.
El desarrollo ágil construye el sistema por medio de aportaciones frecuentes de código que aportan valor
al usuario, es necesario organizar las funcionalidades en bloques y que los cambios siempre surgen de una
necesidad del usuario.
Para cualquier tipo de planificación es indispensable una herramienta de control de versiones, para ello voy a emplear  git
y GitHub como repositorio para hacer un seguimiento de todas las tareas.
He de mencionar que todos los commits deben de hacer referencia a las tareas preestablecidas, por lo tanto,
podremos saber cuando leamos el código hacía que historia de usuario o issue se refiere
y poder observar el desarrollo en conjunto del proyecto

\section{Hitos}
Los hitos pueden definirse como metas a las que debemos de llegar durante el desarrollo del proyecto.
Los hitos son una buena forma de limitar los bloques de trabajo del desarrollo y su principal intención es que
al final de cada hito consigamos un producto mínimamente viable (\textit{PMV}) para que sea independiente e iterativo
Los hitos que se deben de cumplir están reflejados en el GitHub y son los siguientes:

\subsection*{\href{https://github.com/RubenDelgadoPareja/TFG-Triage-Inteligente-Consulta-Medica/milestone/1}{Hito 0: Infraestructura inicial y documentación}}

El primer PMV consiste en alcanzar un repositorio inicial donde comenzar a forjar toda la arquitectura del proyecto con una planificación inicial.
Para aceptar el PMV, el repositorio debe de contener:

\begin{itemize}
    \item{User Journeys.}
    \item{Hitos.}
    \item{Historias de Usuario.}
    \item{Comprobadores de gramática y ortografía.}
    \item{Comprobadores de compilación de LaTeX para la documentación.}
    \item{Documentar las secciones de Motivación, Objetivos y Planificación}
\end{itemize}

\subsection*{\href{https://github.com/RubenDelgadoPareja/TFG-Triage-Inteligente-Consulta-Medica/milestone/7}{Hito 1: Modelado del Problema con DDD}}

Este PMV consiste en modular el dominio del problema para facilitar el diseño e implementación.
Para aceptar el PMV la memoria debe de recoger una sección específica sobre cómo se aplicará el DDD la cual incluya:

\begin{itemize}
    \item {Definición del dominio}
    \item {Definir el lenguaje ubicuo}
    \item {Identificar Entities, Value Objects, Aggregates}
    \item {Definir los Repositories y Services necesarios}
\end{itemize}


\subsection*{\href{https://github.com/RubenDelgadoPareja/TFG-Triage-Inteligente-Consulta-Medica/milestone/2}{Hito 2: Arquitectura básica del sistema de colas de citas}}

Este PMV consiste en diseñar las clases y funciones necesarias para crear citas, almacenarlas en colas y priorizarlas mediante una heurística.

Los criterios de aceptación de este PMV son los tests de cada clase y sus funciones, además de reflejar en la memoria las decisiones tomadas

\section{Historias de Usuarios}
Las historias de usuario son una técnica del desarrollo ágil para representar requisitos utilizando lenguaje común del usuario.
En las historias de usuario se describe desde el punto de vista del usuario, cómo interactúa con el sistema y resuelve una necesidad
Encontraremos las historias de usuario repartidas por los hitos del desarrollo del software. Las historias de usuarios deben seguir el siguiente formato:

\begin{itemize}
    \item{Debemos de identificar el tipo de usuario de la historia.}
    \item{La historia debe de resolver una necesidad del usuario, encontrarse dentro del dominio del problema y expresar el beneficio que consigue el usuario.}
    \item{Cuando se complete la historia de usuario debe de haber una funcionalidad nueva en el producto}
\end{itemize}

\noindent{Se han definido las siguientes historias de usuario:}

\subsection*{\href{https://github.com/RubenDelgadoPareja/TFG-Triage-Inteligente-Consulta-Medica/issues/19}{[HU-01] Triage Inteligente - Como sanitario de urgencias quiero que el triage inteligente asigne turnos de consulta automáticamente por riesgo.}}
Como sanitario de urgencias quiero un sistema de triage inteligente que se encargue de aplicar un cribado de riesgos a los pacientes.
El triage se encarga de evaluar el estado de salud del paciente a través de un formulario dónde se le pregunta al paciente el motivo de la consulta y
síntomas que padece, con esta información el sistema le asignará un riesgo de los cinco niveles establecidos, dependiendo del riesgo el paciente tendrá una demora
específica de tiempo para ser atendido. En caso de que supere el tiempo de demora y no sea atendido se podrá reevaluar el riesgo del paciente.
El triage funcionará todo el día y puede tener supervisión de un sanitario en todo momento, este sanitario podría, según su criterio cambiar el riesgo del paciente.

\subsection*{\href{https://github.com/RubenDelgadoPareja/TFG-Triage-Inteligente-Consulta-Medica/issues/101}{[HU-02] Triage Inteligente - Como sanitario de citología quiero consultar los pacientes con mayores riesgos.}}
Como sanitario de citologías quiero un sistema de triage inteligente que se encargue de aplicar un cribado de riesgos a los pacientes con mayor riesgo.
El triage inteligente en citología estará enfocado a la propia citología, por lo que se tratará de un cuestionario acerca de posibles síntomas alarmantes
que den indicios de la enfermedad. En citología, la demora de consulta es muy alta, por lo que la diagnosis precoz es fundamental para el paciente.

\subsection*{\href{https://github.com/RubenDelgadoPareja/TFG-Triage-Inteligente-Consulta-Medica/issues/5}{[HU-03] Triage Inteligente - Como sanitario de citología quiero modificar el riesgo de un paciente.}}
Como sanitario que trabaja en citología que utiliza el triage inteligente, después de una consulta con un paciente que se le haya diagnosticado de manera precoz una enfermedad grave,
quiero aumentarle el riesgo para las siguientes consultas.

\section{Issues}
Las Issues son simples tareas que hay que realizar para completar los Hitos.
Una historia de Usuario podría ser un conjunto de Issues, aunque en GitHub las he englobado en una sola Issue
Todas las Issues que solucionaré durante el desarrollo para construir el sistema están descritas en GitHub


\section{Control de calidad}

